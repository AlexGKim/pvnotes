\documentclass[11pt, oneside]{article}   	% use "amsart" instead of "article" for AMSLaTeX format
\usepackage{geometry}                		% See geometry.pdf to learn the layout options. There are lots.
\geometry{letterpaper}                   		% ... or a4paper or a5paper or ... 
%\geometry{landscape}                		% Activate for for rotated page geometry
%\usepackage[parfill]{parskip}    		% Activate to begin paragraphs with an empty line rather than an indent
\usepackage{graphicx}				% Use pdf, png, jpg, or eps with pdflatex; use eps in DVI mode
								% TeX will automatically convert eps --> pdf in pdflatex		
\usepackage{amsmath}

\title{Peculiar Velocity Notes}
\author{Alex Kim}
%\date{}							% Activate to display a given date or no date

\begin{document}
\maketitle


\section{Real and Redshift-Space Distances}
Distances  are conveniently expressed
in terms of cosmological redshift.
This distance is denoted as $r$, to signify that this is in real space.
Traditionally we rely on the cosmological redshift to translate to distances
through
\begin{align}
\chi & = \int_{t_e}^{t_o} \frac{dt}{a(t)} \\
&= \int_{0}^{r} \frac{dz}{H(z)}.
\end{align}
For example at small distances we get the familiar $r=H(0) \chi$ Hubble law.

The proper peculiar velocity of a galaxy at position $\mathbf{r}_o$ is
\begin{equation}
\mathbf{v} = \frac{a\,d\mathbf{r}}{dt}= \frac{d\mathbf{r}}{d\tau},
\end{equation}
where $\tau$ is the conformal time.  ``Proper'' means that it is the velocity as
measured by a local observer.

Such a local observer positioned between us and the galaxy receives 
light with  Doppler redshift $z_v$ due to the radial
velocity component $v=\mathbf{v} \cdot  \mathbf{\hat{r}}_o$.
That light getting to us also experiences the cosmological redshift, so  the redshift $s$
arriving to the CMB-frame observer gets is 
\begin{equation}
(1+s)  = (1+r)(1+z_p).
\end{equation}

For low redshift and small peculiar velocities the first order expression is
\begin{equation}
s  = r +v.
\end{equation}
In this domain people often express $r$ and $z_p$ in velocity units.


\section{Standard Candle Information}
The measurements of a standard candle are
\begin{itemize}
\item Angular coordinates RA and Dec.  This direction is denoted as $\mathbf{\hat{r}}_o$.
\item Redshift $s$. In the case of SNe~Ia, the redshift of the host-galaxy is useful for our purposes.
\item Magnitude $m$.
\end{itemize}

Given a cosmological model, we can associate the observed magnitude
with a cosmological distance.  The observed magnitude is
\begin{align}
m & =M + \mu(r) \\
\mu & = 5\log{
	\left(
		\frac{(1+r)\chi(r)}{10 \text pc}
	\right)
},
\end{align}
where $M$ is the absolute magnitude and
 $r$ is the redshift that
we use as a cosmological distance.  
\begin{itemize}
\item The radial component $v$ of the peculiar velocity  can be derived from the above through
the Doppler redshift $z_p$.
\begin{align}
z_p & = \frac{s-r}{1+r}\\
v & = \frac{(1+z_p)^2-1}{(1+z_p)^2+1}.
\end{align}
\end{itemize}

Given a cosmological model, a standard candle with magnitude $m$ is expected
to be at redshift $r$.  Any difference between that and the observer redshift $s$ is a
``peculiar velocity''.
It is useful to consider an alternative point of view.   Given a cosmological model, a standard candle with observed
redshift $s$ is expected to have magnitude $M+\mu(s)$.
Any difference between that and the observer magnitude $m$ is a
``peculiar magnitude'' $\delta m$.  Since $m$ is directly measured, it is convenient to work with
peculiar magnitudes in the analysis.
At low-redshift,  the peculiar velocity and peculiar magnitude
are related by
\begin{align}
\delta m & \approx -\frac{5}{\ln{10}}\frac{v}{aH\chi(r)}\\
& = -\frac{5}{\ln{10}}\frac{(1+r)^2v}{Hd_L(r)}. 
\end{align}



While the data provides both real and redshift-space distances $r$ and $s$ the uncertainty in
$s$, when determined spectroscopically, is exceedingly small compared to the uncertainty in $r$.
Uncertainties in the absolute magnitude  $M$ in particular limit how well $r$ can be
determined.  So the data is most naturally considered in redshift-space.

\section{Velocities and Overdensities Connected Through Fluid Mechanics}

The motion of the cosmological (incompressible) pressureless matter
fluid depends on overdensitites as described by the continuity, Euler
(momentum conservation), and Poisson equations
\begin{equation}
\frac{\partial \delta}{\partial \tau} + \nabla \cdot [(1+\delta)\mathbf{v}]=0
\end{equation}
\begin{equation}
\frac{\partial a \mathbf{v}}{a\partial \tau} + \mathbf{v} \cdot \nabla \mathbf{v} = - \nabla \phi
\end{equation}
\begin{equation}
\nabla^2\phi=\frac{3}{2} \Omega_MH^2a^2\delta.
\end{equation}

The first two  equations to linear order are.
\begin{equation}
\frac{\partial \delta}{\partial \tau} + \nabla \cdot \mathbf{v}=0
\end{equation}
\begin{equation}
\frac{\partial a \mathbf{v}}{a\partial \tau} = - \nabla \phi
\end{equation}

Two important implication of the linear equations are:
\begin{itemize}
\item The peculiar velocity $\mathbf{v}$ today is irrotationlal. The Euler equation gives that the rotational component of $\mathbf{v}$ goes as $a^{-1}$ and so decays away.
\item The density field can be separated
\begin{equation}
\delta (\mathbf{r},\tau)  = D(\tau) \delta (\mathbf{r},\tau_0),
\end{equation}
$D(\tau)$ is called the linear growth factor.  The second order differential equation
gives two solutions for $D$, it is taken as the growing mode.
For calculations $\tau_0$ is often taken to be at the surface of last scattering;
to simplify notation it is omitted from here on.
\item Then the continuity equation can be expressed in terms of $D$
\begin{equation}
Haf\delta(\mathbf{r}) + \nabla \cdot \mathbf{v}=0,
\label{rv:eqn}
\end{equation}
where
\begin{equation}
f \equiv \frac{d \ln{D}}{d\ln{a}}.
\end{equation}
This equation says that the density perturbations at the surface of last scattering $\delta(\mathbf{r})$
is related to the resulting evolved peculiar velocity field with a factor $Haf$.
\end{itemize}

\section{Fourier Space}
In Fourier space, Eq.~\ref{rv:eqn} becomes
\begin{equation}
Haf\delta(\mathbf{k})  + \mathbf{k} \cdot \mathbf{v}(\mathbf{k})=0,
\end{equation}
which has the solution
\begin{equation}
 \mathbf{v}(\mathbf{k},r) = Haf \frac{i\mathbf{k}}{k^2} \delta(\mathbf{k},r).
\end{equation}
The Fourier transform of the projected radial velocity is then
\begin{align}
v(\mathbf{k},r) & = Haf \frac{i\mu}{k} \delta(\mathbf{k},r)\\
&  Haf \frac{i\mu}{k} D(r) \delta(\mathbf{k}),
\label{dvfourier:eqn}
\end{align}
where $\mu = \mathbf{\hat{r}} \cdot \mathbf{\hat{k}}$ that describes the angle between the line of sight and the wave vector
and $\delta(\mathbf{k}) = \delta(\mathbf{k},r=0)$, $D(0)=1$.


For our purposes, we are interested in the statistical properties of the fields.  When the fields are Normally distributed
(in the linear regime) these properties are
captured by the power spectra at $r=0$.
\begin{align}
\langle \delta(\mathbf{k},r)  \delta^{\star}(\mathbf{k'},r') \rangle &  = (2\pi)^3 D(r) D(r') P_{\delta \delta}(\mathbf{k}) \delta(\mathbf{k} -\mathbf{k'})\\ 
\langle \delta(\mathbf{k},r)  v^{\star}(\mathbf{k'},r') \rangle &  = (2\pi)^3 D(r) D(r') P_{\delta v}(\mathbf{k}, \mathbf{\hat{r}'}) \delta(\mathbf{k} -\mathbf{k'})\\ 
\langle v(\mathbf{k},r)  v^{\star}(\mathbf{k'},r') \rangle &  = (2\pi)^3 D(r) D(r') P_{vv}(\mathbf{k},  \mathbf{\hat{r}}, \mathbf{\hat{r}'}) \delta(\mathbf{k} -\mathbf{k'}).
\end{align}

From Eq.~\ref{dvfourier:eqn} the above power spectra are related
\begin{equation}
 P_{\delta \delta}(\mathbf{k})  = -\left( \frac{ik}{\mu' H(r')a(r')f(r')}\right) P_{\delta v}(\mathbf{k}, \mathbf{r'}) \ = k^2 \left( \frac{1}{\mu H(r)a(r)f(r)}\right)
 \left( \frac{1}{\mu' H(r')a(r')f(r')}\right)  P_{vv}(\mathbf{k},\mathbf{r},\mathbf{r'}).
\end{equation}

The above calculations were done in real space, whereas we make our measurements in redshift space.  To first order
the relationship between the real-space and redshift-space velocity
power spectra and the real-space density power spectrum is the same.
\begin{equation}
P^s_{vv}(\mathbf{k}, \mathbf{s}, \mathbf{s'}) \approx P_{vv}(\mathbf{k}, \mathbf{r}, \mathbf{r'}) .
\end{equation}

The  velocity correlation function is then
\begin{align}
\xi_{v v}(\mathbf{r}_i,\mathbf{r}_j)  & =
\langle v(\mathbf{r})  v^{\star}(\mathbf{r'}) \rangle \\
& = D(r_i) D(r_j)  H(r_i) H(r_j) a(r_i) a(r_j) f(r_i) f(r_j)
\int \frac{d^3k}{(2\pi)^3}  \frac{1}{k^2} P_{\delta \delta}(k) 
(\mathbf{\hat{k}} \cdot \mathbf{\hat{r}_i})(\mathbf{\hat{k}} \cdot \mathbf{\hat{r}_j})
e^{-i \mathbf{k}\cdot (\mathbf{r_i}-\mathbf{r_j})} \\
&= \frac{dD_i}{d\tau} \frac{dD_j}{d\tau}   \int \frac{d^3k}{(2\pi)^3}  \frac{1}{k^2}  P_{\delta \delta}(k) (\mathbf{\hat{k}} \cdot \mathbf{\hat{r}_i})(\mathbf{\hat{k}} \cdot \mathbf{\hat{r}_j})  e^{-i \mathbf{k}\cdot (\mathbf{r_i}-\mathbf{r_j})}\\
&= \frac{dD_i}{d\tau} \frac{dD_j}{d\tau}   \int \frac{d^3k}{(2\pi)^3}  \frac{1}{k^2}  P_{\delta \delta}(k) \sum_{l=0}^{\infty}(2l+1) j'_l(kr_i) j'_l(kr_j)
\mathcal{P}_l(\mathbf{\hat{r}}_i \cdot \mathbf{\hat{r}}_j).
\end{align}
The last step takes
\begin{equation}
(\mathbf{\hat{k}} \cdot \mathbf{\hat{r}_j})   e^{i \mathbf{k}\cdot \mathbf{r_j}} = 4\pi \sum_{l,m} i^{l-1}j'_l(kr) Y^{\star}_{lm}(\mathbf{\hat{k}})
Y_{lm}(\mathbf{\hat{r}}_2)
\end{equation}

In terms of peculiar magnitudes for two positions
$\mathbf{r}_i$ and $\mathbf{r}_j$.  To first order 
\begin{align}
\xi_{\delta m \delta m}(\mathbf{r}_i,\mathbf{r}_j)  & \equiv  \langle \delta m_i(\mathbf{r}_i) \delta m_j(\mathbf{r}_j) \rangle\\
& =  \left( \frac{5}{\ln{10}}\right)^2 \frac{(1+r_i)^2}{Hd_L(r_i)} \frac{(1+r_j)^2}{Hd_L(r_j)} \xi_{v v}(\mathbf{r}_i,\mathbf{r}_j)  .
\end{align}

While the above derivation is based on $\chi(r)$ , the difference when using 
 $\chi(s)$  is of second order.
\end{document}  