\documentclass[11pt, oneside]{article}   	% use "amsart" instead of "article" for AMSLaTeX format
\usepackage{geometry}                		% See geometry.pdf to learn the layout options. There are lots.
\geometry{letterpaper}                   		% ... or a4paper or a5paper or ... 
%\geometry{landscape}                		% Activate for for rotated page geometry
%\usepackage[parfill]{parskip}    		% Activate to begin paragraphs with an empty line rather than an indent
\usepackage{graphicx}				% Use pdf, png, jpg, or eps with pdflatex; use eps in DVI mode
								% TeX will automatically convert eps --> pdf in pdflatex		
\usepackage{amsmath,amssymb}
\usepackage{verbatim}

\title{Peculiar Velocity Notes}
\author{Alex Kim}
%\date{}							% Activate to display a given date or no date

\begin{document}
\maketitle


\section{Real and Redshift-Space Distances}
\label{distances:sec}
Distances  are conveniently expressed
in terms of cosmological redshift.
This distance is denoted as $r$, to signify that this is in real space.
A distance expressed as a cosmological redshift  translates to the conformal distance $\chi$
through
\begin{align}
\chi & = \int_{t_e}^{t_o} \frac{dt}{a(t)} \\
&= \int_{0}^{r} \frac{dz}{H(z)}.
\end{align}
For example at small distances we get the familiar $r=H(0) \chi$ Hubble law.

The proper peculiar velocity of a galaxy at position $\mathbf{r}_o$ is
\begin{equation}
\mathbf{v} = \frac{a\,d\mathbf{r}}{dt}= \frac{d\mathbf{r}}{d\tau},
\end{equation}
where $\tau$ is the conformal time.  ``Proper'' means that it is the velocity as
measured by a local observer.

A galaxy photon moving in the direction $-\mathbf{\hat{r}}_o$ will have
a   Doppler redshift $z_v$ due to the radial
velocity component $v=\mathbf{v} \cdot  \mathbf{\hat{r}}_o$ for the observer at $\mathbf{r}_o$.
That photon undergoes a further cosmological redshift so the redshift $s$
arriving to us (neglecting our peculiar motion, i.e.\ in the CMB frame) is 
\begin{equation}
(1+s)  = (1+r)(1+z_p).
\end{equation}
For low redshift and small peculiar velocities the first order expression is
\begin{equation}
s  = r +v.
\end{equation}

The standard is for distances to be given in units $[\text{Mpc}\,h^{-1}]$.  Multiplying
by 100 converts from these units  to $[\text{km}\,\text{s}^{-1}]$.


\section{Standard Candle Information}
The measurements of a standard candle are
\begin{itemize}
\item Angular coordinates RA and Dec.  This direction is denoted as $\mathbf{\hat{r}}_o$.
\item Redshift $s$. In the case of SNe~Ia, the redshift of the host-galaxy is useful for our purposes.
\item Magnitude $m$.
\end{itemize}

An important parameter that can derived from the data assuming a model is the cosmological distance $r$.
Given a cosmological model, the observed magnitude of a standard candle is
\begin{align}
m & =M + \mu(r; H_0, \Omega,\ldots) \\
& = (M -5\log{H_0}) + (5\log{H_0}+ \mu(r; H_0, \Omega,\ldots))\\
& = \mathcal{M} +  (\mu+5\log{H_0})(r;  \Omega,\ldots),
\label{mag:eqn}
\end{align}
where $\mathcal{M} = M -5\log{H_0}$, in Friedmann cosmologies
\begin{align}
\mu & = 5\log{
	\left(
		\frac{(1+s_{\odot})\chi(r; H_0, \Omega,\ldots)}{10 \text pc}
	\right)
},
\end{align}
and the combination $\mu+5\log{H_0}$ is independent of $H_0$.  (The $s_{\odot}$ is a reminder that the the galaxy and the sun's peculiar
velocities contribute to time dilation and redshift.)
It happens that a large set of observational 
data gives tight constraints on $\mathcal{M}$, whereas uncertainties on $M$ and $H_0$ individually are
rather large.   Eq.~\ref{mag:eqn} gives the transformation
between the observed $m$ to the cosmological distances $r(m; \mathcal{M}, \Omega,\ldots)$, which does not explicitly depend on $H_0$.
At low redshifts where the linear Hubble law applies
\begin{equation}
r(\mathcal{M}) \approx 10^
{
\frac{(m-\mathcal{M})}{5}
}.
\end{equation}
Although not always notated as such, keep in mind that $r$ depends on $\mathcal{M}$ and the cosmological
parameters $\Omega, \ldots$, but not $M$ or $H_0$.

Any difference between the observed $s$ and cosmological $r$ redshifts can be described as
a
``peculiar velocity''.
The radial component $v$ of the peculiar velocity  can be derived from the above through
the Doppler redshift $z_p$,
\begin{align}
z_p & = \frac{s-r}{1+r},\\
v & = \frac{(1+z_p)^2-1}{(1+z_p)^2+1}.
\end{align}
Note that in the linear regime the only parameter-dependence of the peculiar velocity is on $\mathcal{M}$, which is well
constrained from external observations.
The standard candles we are thinking about are at low-redshift, so it is useful to think of the velocities as being
in velocity units.
Peculiar velocities are useful to consider because theory (fluid dynamics) directly relates velocity flows with density perturbations
as will be discussed in \S\ref{fluid:sec}.


It is useful to consider an alternative point of view, where 
any difference between the observed $m$ and model-expected $\mathcal{M} + (\mu(s)+5\log{H_0})$ magnitudes
can be described as
a ``peculiar magnitude''  $\delta m$.  A peculiar velocity $v$ can be alternatively be attributed to 
the peculiar magnitude via
\begin{align}
\delta m & = -\frac{5}{\ln{10}}\frac{(1+r)^2v}{Hd_L(r)}. 
\end{align}
Since $m$ is directly measured, it is convenient to work with
peculiar magnitudes in the analysis, while transforming theory velocity predictions into peculiar magnitude predictions.


While the data provides both real and redshift-space distances $r$ and $s$ the uncertainty in
$s$, when determined spectroscopically, is exceedingly small compared to the uncertainty in $r$.
Uncertainties in the absolute magnitude  $M$ in particular limit how well $r$ can be
determined.  So the data is most naturally considered in redshift-space.

\section{Relation Between Velocities and Overdensities}
\label{fluid:sec}
The motion of the cosmological (incompressible) pressureless matter
fluid depends on overdensitites as described by the continuity, Euler
(momentum conservation), and Poisson equations
\begin{equation}
\frac{\partial \delta}{\partial \tau} + \nabla \cdot [(1+\delta)\mathbf{v}]=0
\end{equation}
\begin{equation}
\frac{\partial a \mathbf{v}}{a\partial \tau} + \mathbf{v} \cdot \nabla \mathbf{v} = - \nabla \phi
\end{equation}
\begin{equation}
\nabla^2\phi=\frac{3}{2} \Omega_MH^2a^2\delta.
\end{equation}
The spatial derivatives are in comoving coordinates  $\mathbf{x}$ with length units.

\subsection{Linear Order in Velocity}
The first two  equations to linear order are.
\begin{equation}
\frac{\partial \delta}{\partial \tau} + \nabla \cdot \mathbf{v}=0
\end{equation}
\begin{equation}
\frac{\partial a \mathbf{v}}{a\partial \tau} = - \nabla \phi
\end{equation}

Two important implication of the linear equations are:
\begin{itemize}
\item The peculiar velocity $\mathbf{v}$ today is irrotationlal. The Euler equation gives that the rotational component of $\mathbf{v}$ goes as $a^{-1}$ and so decays away.
\item The density field can be separated
\begin{equation}
\delta (\mathbf{x},\tau)  = D(\tau) \delta (\mathbf{x},\tau_0),
\end{equation}
$D(\tau)$ is  the linear growth factor.  $D$ has two solutions, we consider only the growing solution.
The notation  $\delta (\mathbf{x})$ without the time dependence is understood to be the density field at
some fiducial $\tau_0$.  Often we considered the density field today or at the surface of last scattering.
\item The continuity equation can be expressed in terms of $D$
\begin{equation}
Haf\delta(\mathbf{x}) + \nabla \cdot \mathbf{v}=0,
\label{rv:eqn}
\end{equation}
where
\begin{equation}
f \equiv \frac{d \ln{D}}{d\ln{a}}.
\label{f:eqn}
\end{equation}
This equation says that at any given $\tau$, density perturbations  $\delta(\mathbf{r})$
are related to the peculiar velocity field with a factor $Haf$.
\end{itemize}

In Fourier space, Eq.~\ref{rv:eqn} becomes
\begin{equation}
Haf\delta(\mathbf{k})  + \mathbf{k} \cdot \mathbf{v}(\mathbf{k})=0,
\end{equation}
which has the solution
\begin{equation}
 \mathbf{v}(\mathbf{k},x) = Haf \frac{i\mathbf{k}}{k^2} \delta(\mathbf{k},x).
\end{equation}
The Fourier transform of the projected radial velocity is then
\begin{align}
v(\mathbf{k},x) & = Haf \frac{i\mu}{k} \delta(\mathbf{k},x)\\
&=  Haf \frac{i\mu}{k} D(r) \delta(\mathbf{k}),
\label{dvfourier_Mpc:eqn}
\end{align}
where $\mu = \mathbf{\hat{r}} \cdot \mathbf{\hat{k}}$ that describes the angle between the line of sight and the wave vector
and the overdensity is at some reference redshift $r_0$ such that $\delta(\mathbf{k}) = \delta(\mathbf{k},r=r_0)$, $D(r_0)=1$.

The above  result is shown in textbooks, but is not actually what
we want to use.
The above equations  are set up such that distances are measured in
comoving length coordinates (e.g.\ Mpc) with the density and velocity fields functions of $\mathbf{x}$.
However, since we observe redshifts more often than distances,
the standard in the field is to use length coordinates with units Mpc\,$h^{-1}$, i.e.\ $\mathbf{r} = h^{-1}\mathbf{x}$.
Then $\nabla_r = h \nabla_x$,
and the continuity equation becomes
\begin{equation}
Haf\delta(\mathbf{r}) + h \nabla \cdot \mathbf{v}=0,
\label{rv_h:eqn}
\end{equation}
and so
\begin{align}
v(\mathbf{k},x 
&=  h^{-1} Haf\frac{i\mu}{k} D(r) \delta(\mathbf{k})\\
&=  100  \frac{H}{H_0}af \frac{i\mu}{k} D(r) \delta(\mathbf{k}),
\label{dvfourier_Mpchinv:eqn}
\end{align}
with the velocities in $[\text{km}\,\text{s}^{-1}]$ units.
Working in redshift space removes dependence on the value of the Hubble constant.

\begin{comment}
However we take the convention of expressing distances in terms of redshift  with the density and velocity fields functions of $\mathbf{r}$.
Since $\nabla_x = H_0 \nabla_r$,
\begin{equation}
Haf\delta(\mathbf{k})  + H_0 \mathbf{k} \cdot \mathbf{v}(\mathbf{k})=0,
\end{equation}
which has the solution
\begin{equation}
 \mathbf{v}(\mathbf{k},r) = \frac{H}{H_0}af \frac{i\mathbf{k}}{k^2} \delta(\mathbf{k},r).
\end{equation}
The Fourier transform of the projected radial velocity is then
\begin{align}
v(\mathbf{k},r) & = \frac{H}{H_0}af \frac{i\mu}{k} \delta(\mathbf{k},r)\\
&=  \frac{H}{H_0}af \frac{i\mu}{k} D(r) \delta(\mathbf{k}),
\label{dvfourier:eqn}
\end{align}
\end{comment}

\subsection{Linear Order in Momentum}
Defining the momentum field as
\begin{equation}
\mathbf{p} = (1+\delta)\mathbf{v}
\end{equation}
we see that the continuity equation gives 
\begin{equation}
\frac{\partial \delta}{\partial \tau} + \nabla \cdot \mathbf{p}=0
\end{equation}



\section{Statistical Measure}

We are interested in the statistical properties of peculiar velocities, in particular their correlation function or equivalently
their power spectrum.  Although we measure the peculiar velocity of individual galaxies, combining velocities of the galaxies
actually observed, whose locations are weighted by density, leads to a statistical measurement of the peculiar momentum field.
This is easily seen in power-spectrum space, where the Fourier integral over volume is approximated by a sum
over the observed galaxies.  The observed galaxies come from a biased distribution. Viewed the other way, a straight
sum over the galaxies means that the Fourier transform is done over velocities weighted by density.


For our purposes, we are interested in the statistical properties of the fields.  When the fields are Normally distributed
(in the linear regime) these properties are
captured by the power spectra at the fiducial $r_0$.
\begin{align}
\langle \delta(\mathbf{k},r)  \delta^{\star}(\mathbf{k'},r') \rangle &  = (2\pi)^3 D(r) D(r') P_{\delta \delta}(\mathbf{k}) \delta(\mathbf{k} -\mathbf{k'})\\ 
\langle \delta(\mathbf{k},r)  v^{\star}(\mathbf{k'},r') \rangle &  = (2\pi)^3 D(r) D(r') P_{\delta v}(\mathbf{k}, \mathbf{\hat{r}'}) \delta(\mathbf{k} -\mathbf{k'})\\ 
\langle v(\mathbf{k},r)  v^{\star}(\mathbf{k'},r') \rangle &  = (2\pi)^3 D(r) D(r') P_{vv}(\mathbf{k},  \mathbf{\hat{r}}, \mathbf{\hat{r}'}) \delta(\mathbf{k} -\mathbf{k'}).
\end{align}

From Eq.~\ref{dvfourier_Mpchinv:eqn} the above power spectra are related
%\begin{equation}
% P_{\delta \delta}(\mathbf{k})  = -\left( \frac{ikH_0}{\mu' H(r')a(r')f(r')}\right) P_{\delta v}(\mathbf{k}, \mathbf{r'}) \ = k^2 \left( \frac{H_0}{\mu H(r)a(r)f(r)}\right)
% \left( \frac{H_0}{\mu' H(r')a(r')f(r')}\right)  P_{vv}(\mathbf{k},\mathbf{r},\mathbf{r'}).
%\end{equation}
\begin{equation}
 P_{vv}(\mathbf{k})  = k^{-2} \left( \frac{\mu H(r)a(r)f(r)}{H_0}\right)
 \left( \frac{\mu' H(r')a(r')f(r')}{H_0}\right)  P_{\delta\delta}(\mathbf{k},\mathbf{r},\mathbf{r'}).
\end{equation}
\begin{equation}
 P_{\delta v}(\mathbf{k})  = k^{-1}
 \left( \frac{\mu' H(r')a(r')f(r')}{H_0}\right)  P_{\delta\delta}(\mathbf{k},\mathbf{r},\mathbf{r'}).
\end{equation}


The  velocity correlation function is then
\begin{align}
\xi_{v v}(\mathbf{r}_i,\mathbf{r}_j)  & =
\langle v(\mathbf{r})  v^{\star}(\mathbf{r'}) \rangle \\
& = D(r_i) D(r_j)  \frac{H(r_i)}{H_0} \frac{H(r_j)}{H_0} a(r_i) a(r_j) f(r_i) f(r_j)
\int \frac{d^3k}{(2\pi)^3}  \frac{1}{k^2} P_{\delta \delta}(k) 
(\mathbf{\hat{k}} \cdot \mathbf{\hat{r}_i})(\mathbf{\hat{k}} \cdot \mathbf{\hat{r}_j})
e^{-i \mathbf{k}\cdot (\mathbf{r_i}-\mathbf{r_j})} \\
&= H_0^{-2}\frac{dD_i}{d\tau} \frac{dD_j}{d\tau}   \int \frac{d^3k}{(2\pi)^3}  \frac{1}{k^2}  P_{\delta \delta}(k) (\mathbf{\hat{k}} \cdot \mathbf{\hat{r}_i})(\mathbf{\hat{k}} \cdot \mathbf{\hat{r}_j})  e^{-i \mathbf{k}\cdot (\mathbf{r_i}-\mathbf{r_j})}\\
&= H_0^{-2} \frac{dD_i}{d\tau} \frac{dD_j}{d\tau}   \int \frac{d^3k}{(2\pi)^3}  \frac{1}{k^2}  P_{\delta \delta}(k) \sum_{l=0}^{\infty}(2l+1) j'_l(kr_i) j'_l(kr_j)
\mathcal{P}_l(\mathbf{\hat{r}}_i \cdot \mathbf{\hat{r}}_j).
\end{align}
The last step uses the decomposition
\begin{equation}
(\mathbf{\hat{k}} \cdot \mathbf{\hat{r}_j})   e^{i \mathbf{k}\cdot \mathbf{r_j}} = 4\pi \sum_{l,m} i^{l-1}j'_l(kr) Y^{\star}_{lm}(\mathbf{\hat{k}})
Y_{lm}(\mathbf{\hat{r}}_j).
\end{equation}

The correlation for the peculiar magnitudes at two positions
$\mathbf{r}_i$ and $\mathbf{r}_j$ is to first order 
\begin{align}
\xi_{\delta m \delta m}(\mathbf{r}_i,\mathbf{r}_j)  & \equiv  \langle \delta m_i(\mathbf{r}_i) \delta m_j(\mathbf{r}_j) \rangle\\
& =  \left( \frac{5}{\ln{10}}\right)^2  H_0^{-2}\frac{(1+r_i)^2}{H(r_i)d_L(r_i)} \frac{(1+r_j)^2}{H(r_j)d_L(r_j)} \xi_{v v}(\mathbf{r}_i,\mathbf{r}_j)  \\
& =  \left( \frac{5}{\ln{10}}\right)^2 \frac{(1+r_i)f(r_i)D(r_i)}{H_0d_L(r_i)} \frac{(1+r_j)f(r_j)D(r_j)}{H_0d_L(r_j)}  \\
& \times  \int \frac{d^3k}{(2\pi)^3}  \frac{1}{k^2}  P_{\delta \delta}(k) \sum_{l=0}^{\infty}(2l+1) j'_l(kr_i) j'_l(kr_j)
\mathcal{P}_l(\mathbf{\hat{r}}_i \cdot \mathbf{\hat{r}}_j).
\end{align}

There is the tradition of expressing the normalization of the power spectrum in terms of
the variance in the mass overdensity in 8 Mpc spheres 
\begin{align}
\sigma^2_8 (r)& = \frac{1}{2\pi^2} \int_0^\infty  k^2 D^2(r)  P_{\delta \delta}(k) \tilde{W_8}(k)^2 dk
\label{sigma_8:eqn}
\end{align}
where $\tilde{W_8}$ is the Fourier transform of the comoving 8 Mpc top-hat filter.
As equivalent normalization factors, $\sigma_8$ can be used in place of $D$.

The CMB provides an excellent measure of $P_{\delta \delta}(k)$ in the same velocity-as-length units as adopted
for the measurements.  So we take that as a given.

We see that a
peculiar magnitude measurement at redshift $r$ is sensitive to the combination
\begin{equation}
\frac{f(r(\mathcal{M}, \Omega,\ldots)) \sigma_8(r(\mathcal{M}, \Omega,\ldots))}{H_0 d_L(r(\mathcal{M}, \Omega,\ldots); H_0, \Omega,\ldots)}=
\frac{f(r(\mathcal{M}, \Omega,\ldots)) \sigma_8(r(\mathcal{M}, \Omega,\ldots))}{(H_0 d_L)(r(\mathcal{M}, \Omega,\ldots); \Omega,\ldots)},
\end{equation}
given that the combination $H_0 d_L$ is independent of the Hubble constant.
At low redshift where we take a linear Hubble law as given, we find that our peculiar velocities derived from a standard candle and
a fiducial power spectrum, say that inferred by the CMB,
give a measurement of $f(r)\sigma_8(r)$ with a dependence on the parameter $\mathcal{M}$.
 
\section{Redshift Space}
The above calculations were done in real space, whereas we make our measurements in redshift space.  To first order
the relationship between the real-space and redshift-space velocity
power spectra and the real-space density power spectrum is the same.
\begin{equation}
P^s_{vv}(\mathbf{k}, \mathbf{s}, \mathbf{s'}) \approx P_{vv}(\mathbf{k}, \mathbf{s}, \mathbf{s'}) .
\end{equation}
When using this approximation, we want to avoid regimes where the real- and velocity-space distances may be significantly different,
e.g.\ where $s \lesssim 5000$ km s$^{-1}$.

\section{Testing Gravity}
Theory gives a prediction for the expected values of $(f(r) \sigma_8(r))$.
Introducing the parameter $\gamma$ such that
\begin{equation}
f= \Omega_M^\gamma,
\end{equation}
it is known that General Relativity is well described by $\gamma \approx 0.55$ and that other gravity models can 
be described by different values of $\gamma$;  $D(r)$ can  be determined using Eq.~\ref{f:eqn};
$\sigma_8(r)$ can be determined using Eq.~\ref{sigma_8:eqn}.

\section{Fisher Calculation}
Let us consider a simple model.
The growth of structure is
parameterized by $\gamma$ and $\Omega_{M0}$, from which 
$f(a)$ and $D(a)$ are calculated,
or alternatively by effective
$bD$ and $fD$.
Galaxy bias is parameterized by $b$.
Taking the universe to be periodic
on a very large scale $L$, 
the overdensity field at
can be
described by a finite set
of values
at grid of points in 3-dimensional
$k$-space, $\delta_{\mathbf{k}} = D(a) |\delta^{\text{CMB}}_{\mathbf{k}}| e^{i\phi_{\mathbf{k}}}$,
which is parameterized by
$|\delta^{\text{CMB}}_{\mathbf{k}}|$
and $\phi_{\mathbf{k}}$ up to some
$k_{\text{max}}$.  The galaxy
overdensity field is
\begin{equation}
    \delta^g(\mathbf{x}) = bD \sum_{\mathbf{k}} |\delta^{\text{CMB}}_\mathbf{k}| \cos{\left(\frac{2\pi \mathbf{k} \cdot \mathbf{x}}{L} + \phi_\mathbf{k}\right)}.
\end{equation}
and the peculiar magnitude field is
\begin{align}
       m(\mathbf{x})
       & = \frac{5}{\ln{10}}\frac{fD}{ad_L} \sum_{\mathbf{k}}
       \frac{\mu}{k} |\delta^{\text{CMB}}_\mathbf{k}| \sin{\left(\frac{2\pi \mathbf{k} \cdot \mathbf{x}}{L} + \phi_\mathbf{k}\right)}.
\end{align}

Assuming Gaussian measurement
errors, the likelihood for
the observations $\delta_g$
and $m$ is
\begin{align}
    L& =\sum_\mathbf{k} L_k\\
    L_\mathbf{k}& = \int d|\delta^{\text{CMB}}_\mathbf{k}| d\phi_\mathbf{k}
    \mathcal{N}(\delta_g-\delta_g(x),\sigma_{\delta_g})
    \mathcal{N}(m-m(x),\sigma_m)
    \mathcal{N}(|\delta^{\text{CMB}}_\mathbf{k}|,P^{\text{CMB}}(k))
\end{align}

\begin{align}
     \mathcal{N}(\delta_g-\delta_g(x),\sigma_{\delta_g})& \propto
         \exp{
         \left(
         -
         \frac{\left(\delta_g-bD \sum_{\mathbf{k}} |\delta^{\text{CMB}}_\mathbf{k}| \cos{\left(\frac{2\pi \mathbf{k} \cdot \mathbf{x}}{L} + \phi_\mathbf{k}\right)}\right)^2}{2\sigma^2_{\delta_g}}
         \right)}
\end{align}

At the time of the CMB, the volume
around us had density fluctuations
that could be described
by a grid of points in 3-dimensional
$k$-space, $\delta^{\text{CMB}}_{\mathbf{k}}$,  where 
\begin{align}
    |\delta^{\text{CMB}}_{\mathbf{k}}| & \sim \mathcal{N}\left(0,P^{\text{CMB}}_{\delta\delta}\left(\frac{k}{L}\right)\right).
\end{align}
In linear theory the overdensity
field
today $\delta_{\mathbf{k}} = |\delta_{\mathbf{k}}|e^{i\phi_{\mathbf{k}}}$ is then described
by
\begin{align}
    |\delta_{\mathbf{k}}| & = D |\delta^{\text{CMB}}_{\mathbf{k}}|
\\
    \phi_{\mathbf{k}} & \sim \mathcal{U}(0,2\pi).
\end{align}
such that the density field in real space is
\begin{equation}
    \delta(\mathbf{x}) = \sum_{\mathbf{k}} |\delta_\mathbf{k}| \cos{\left(\frac{2\pi \mathbf{k} \cdot \mathbf{x}}{L} + \phi_\mathbf{k}\right)}.
\end{equation}
Neglecting here redshift space
distortions
\begin{equation}
    \delta^g(\mathbf{x}) = bD \sum_{\mathbf{k}} |\delta^{\text{CMB}}_\mathbf{k}| \cos{\left(\frac{2\pi \mathbf{k} \cdot \mathbf{x}}{L} + \phi_\mathbf{k}\right)}.
\end{equation}

In linear theory the peculiar
velocity field along the line of
sight is (slightly wrong in
the middle)
\begin{align}
   v_\mathbf{k} & = Haf \frac{i\mu}{k} \delta_\mathbf{k}\\
       v(\mathbf{x}) & = -\sum_{\mathbf{k}} |v_\mathbf{k}| \sin{\left(\frac{2\pi \mathbf{k} \cdot \mathbf{x}}{L} + \phi_\mathbf{k}\right)}\\
       & = -HafD\sum_{\mathbf{k}}
       \frac{\mu}{k} |\delta^{\text{CMB}}_\mathbf{k}| \sin{\left(\frac{2\pi \mathbf{k} \cdot \mathbf{x}}{L} + \phi_\mathbf{k}\right)}.
\end{align}

So now we see how our observables
$\delta^g$ and $v$ are related to
our model parameters for
the underlying overdensity field
$|\delta^{\text{CMB}}_\mathbf{k}|$
and $\phi_\mathbf{k}$,
and the other parameters of interest

\begin{align}
    P_{\delta \delta}(k; a) &= D^2(a) P_{\delta \delta}(k; a=\text{CMB})\\
    \delta(\mathbf{k}) & \sim \mathcal{N}\left(P_{\delta \delta}\left(k;a\right)\right)\\
    \delta^g(\mathbf{k}) & =b\delta(\mathbf{k}) \\
    \delta^g(\mathbf{r}) & =\text{FT}(\delta^g(\mathbf{k}))\\
    v(\mathbf{k}) & = Haf \frac{i\mu}{k} \delta(\mathbf{k})\\ 
    v(\mathbf{r}) & = \text{FT}(v(\mathbf{k}))\\
    n^g(\mathbf{r})dV &  \sim \text{Poisson}(\bar{n}\delta^g(\mathbf{r})dV)\\
    v_{\text{GW}} & \sim \mathcal{N}(v(\mathbf{r}),\sigma_v)
\end{align}
The diagonal terms of the matrix
in Equation~\ref{cov:eq} contain
the autocorrelations in
galaxy counts and peculiar velocities for a single $k$-mode
measured in Fourier space as
predicted
from the energy correlations measured
at the CMB.  We include in the velocity shot-noise a 300~km\,s$^{-1}$
term that represents sources
of peculiar velocity that are not represented in our model e.g.\ contributions
from  $k$-modes excluded
in our calculations.  It does not include errors
that are in common between galaxy
counts and velocities, for example
differences between CAMB predictions
and the true overdensity field today.

\section{Correlation Function}
Gorski identified statistics $\psi_1$ and $\psi_2$ based on pairs of observed radial peculiar velocities $u$
\begin{align}
\psi_1(r) & = \frac{\sum u_1 u_2 \cos{\theta_{12}}}{\sum \cos^2{\theta_{12}}} \\
\psi_2(r) & = \frac{\sum u_1 u_2 \cos{\theta_{1}}\cos{\theta_2}}{\sum \cos{\theta_{12} \cos{\theta_{1}} \cos{\theta_{2}}}} 
\end{align}
where the $\theta_{12}$ is the angle between $\vec{r}_1$ and $\vec{r}_1$, and $\theta_i$ is the angle between $\vec{r}_i$ and $\vec{r}_2-\vec{r}_1$.

Gorski takes these functions in the continuous field limit of large $N$, where the summation is replaced with integrals over the two positions,
delta-functions to pick out those pairs with the correct $r$, and the probability of selecting a galaxy, which includes the survey
selection function and intrinsic density contrast.  If one then neglects the intrinsic density contrast, which means the
absence of strong clustering of the galaxies, the expectation values
of $\psi$'s can be expressed in a convenient form.

The $\langle u_1 u_2 \rangle$ contribution comes as follows:
The expectation of the
two-point correlation tensor of the peculiar velocity field at positions $1$ and $2$ averaged over all possible realizations
is denoted as
\begin{equation}
\Psi_{ij}(\vec{r}_1, \vec{r}_2)=\langle v_i(\vec{r}_1) v_j(\vec{r}_2)\rangle.
\end{equation}
For an irrotational, homogeneous, and isotropic universe this can be expressed
\begin{equation}
\Psi_{ij}(r)=[\Psi_\parallel(r)-\Psi_\perp(r)] \hat{r}_{i} \hat{r}_{j} + \Psi_\perp(r) \delta^K_{ij}
\end{equation}
where $\vec{r}=|\vec{r}_1- \vec{r}_2|$.
[Turner et al.\ has a typo.]
The  radial peculiar velocity correlation can be determined from the above by applying some dot products
\begin{equation}
\langle u_1 u_2 \rangle = \Psi_\perp \cos{\theta_{12}} + [\Psi_\parallel - \Psi_\perp] \cos{\theta_1} \cos{\theta_2}.
\end{equation}
$\Psi_\parallel(r)$ and $\Psi_\perp(r)$ can be calculated given a power spectrum.

By neglecting the density contrast, all the other terms are geometric and do not depend on ensemble averaging.
Introducing
\begin{align}
\mathcal{A}(r) &=\frac{ \int_V d^3r_1 d^3r_2 \delta_D(|r_2-r_1|-r) \phi(r_1)\phi(r_2)  \cos{\theta_1}\cos{\theta_2}\cos{\theta_{12}}}
{ \int_V d^3r_1 d^3r_2 \delta_D(|r_2-r_1|-r) \phi(r_1)\phi(r_2)  \cos^2{\theta_{12}}}\\
\mathcal{B}(r) &= \frac{ \int_V d^3r_1 d^3r_2 \delta_D(|r_2-r_1|-r) \phi(r_1)\phi(r_2)  \cos^2{\theta_1}\cos^2{\theta_2}}
{ \int_V d^3r_1 d^3r_2 \delta_D(|r_2-r_1|-r) \phi(r_1)\phi(r_2)  \cos{\theta_1}\cos{\theta_2}\cos{\theta_{12}}}
\end{align}
the expectation values of the observables are
\begin{align}
\langle \psi_1(r) \rangle &= \mathcal{A}(r) \Psi_{\perp}(r) + [1-\mathcal{A}(r)] \Psi_{\parallel} (r)\\
\langle \psi_2(r) \rangle &= \mathcal{B}(r) \Psi_{\perp}(r) + [1-\mathcal{B}(r)] \Psi_{\parallel} (r)
\end{align}
where $\phi$ is the survey selection function.
Note that in analysis in the literature, $\mathcal{A}$ and $\mathcal{B}$ are approximated using the observed data.

To summarize, the observer statistics $\psi_1$ and $\psi_2$ in some limit and with some assumptions can be
expressed as a combination of functions of the underlying velocity correlations and functions that describe the survey
geometry.

\end{document}  